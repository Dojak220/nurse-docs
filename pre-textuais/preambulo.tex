%
% ********** Página de assinaturas
%

\begin{titlepage}

\begin{center}

\LARGE

\textbf{Nurse: aplicação mobile para controle dos dados de vacinação e aumento de produtividade de profissionais da saúde}

\vfill

\Large

\textbf{Dorgival da Rocha Filho}

\end{center}

\vfill

% O \noindent é para eliminar a tabulação inicial que normalmente é
% colocada na primeira frase dos parágrafos
\noindent
% Descomente a opção que se aplica ao seu caso
% Note que propostas de tema de qualificação nunca têm preâmbulo.
Monografia aprovada em 09 de dezembro de 2022, pela banca examinadora composta
pelos seguintes membros:

% Os nomes dos membros da banca examinadora devem ser listados
% na seguinte ordem: orientador, co-orientador (caso haja),
% examinadores externos, examinadores internos. Dentro de uma mesma
% categoria, por ordem alfabética

\begin{center}

\vspace{1.5cm}\rule{0.95\linewidth}{1pt}
\parbox{0.9\linewidth}{%
Prof. Dr. Itamir de Morais Barroca Filho (orientador) \dotfill\ DTI-IMD/UFRN}

% \vspace{1.5cm}\rule{0.95\linewidth}{1pt}
% \parbox{0.9\linewidth}{%
% Prof. Dr. YYYYY (co-orientador) \dotfill\ MCA/UFRN}

\vspace{1.5cm}\rule{0.95\linewidth}{1pt}
\parbox{0.9\linewidth}{%
Prof. Dr. André Morais Gurgel \dotfill\ DEPAD/CCSA/UFRN}

\vspace{1.5cm}\rule{0.95\linewidth}{1pt}
\parbox{0.9\linewidth}{%
Prof. Dr. Jean Mário Moreira de Lima \dotfill\ IMD/UFRN}

\end{center}

\end{titlepage}

%
% ********** Dedicatória
%

% A dedicatória não é obrigatória. Se você tem alguém ou algo que teve
% uma importância fundamental ao longo do seu curso, pode dedicar a ele(a)
% este trabalho. Geralmente não se faz dedicatória a várias pessoas: para
% isso existe a seção de agradecimentos.
% Se não quiser dedicatória, basta excluir o texto entre
% \begin{titlepage} e \end{titlepage}

\begin{titlepage}

\vspace*{\fill}

\hfill
\begin{minipage}{0.5\linewidth}
\begin{flushright}
\large\it
Aos meus pais, que não apenas acreditaram, mas reforçaram e defenderam, todos os dias, que a educação é a melhor forma de gerar mudanças na vida e na sociedade.
\end{flushright}
\end{minipage}

\vspace*{\fill}

\end{titlepage}

%
% ********** Agradecimentos
%

% Os agradecimentos não são obrigatórios. Se existem pessoas que lhe
% ajudaram ao longo do seu curso, pode incluir um agradecimento.
% Se não quiser agradecimentos, basta excluir o texto após \chapter*{...}

\chapter*{Agradecimentos}
\thispagestyle{empty}

\begin{trivlist}  \itemsep 2ex

  \item Eu quero agradecer primeiramente a Deus, pois Ele é o centro e a Ele eu devo tudo.

  \item Eu sou o conjunto de todas vivências que tive, pois foram elas que me ensinaram novas coisas ou que me mostraram novas perspectivas sobre aquilo que eu já conhecia; de todas as pessoas que conheci, pois com cada uma delas eu aprendi e ensinei, entreguei e recebi e guardo na memória pequenos e grandes momentos que marcaram a minha vida; de todas as oportunidades que aceitei e também as que recusei, as quais me mostraram os inúmeros caminhos possíveis e a convicção de que nunca existirá o mais correto ou o único. Por tudo isso, momentos bons e de aprendizados, eu agradeço.
  
  \item Entre todas as vivências, pessoas e oportunidade, algumas se destacam e a elas eu quero dedicar esse agradecimento.
  
  \item À minha família, que faz jus ao significado dessa palavra. Eu nunca tive medo de me mudar para cursar em outro estado ou de arriscar naquilo que não era garantido porque eu sempre soube que, independentemente do resultado, eu poderia contar com meu pai Dorgival, minha mãe Maninha e meu irmão Wilsin. Sempre soube que quando eu precisasse, teria um lugar pra voltar e pessoas que me amam para me acolher e confiei e confio minha vida a eles. Isso se estende à toda a minha família. Eu amo todos vocês!
  
  \item Ao meu pai, que sempre esteve presente para me dar conselhos, me ensinar a ser um homem responsável, bondoso e corajoso. Ele sempre colocou seus filhos à frente de seus interesses pessoais e por muitas vezes enfrentou grandes desafios e encontrou forças em Deus para que nunca faltasse nada em nossas vidas. Não só isso, mas nunca mediu esforços para que eu tivesse uma vida confortável em Natal, até mesmo nos momentos de voltar pra casa, quando ele fazia questão de me buscar.
  
  \item À minha mãe, que sempre foi nosso porto seguro e minha grande amiga, em quem confio minha vida de olhos fechados e que esteve presente sempre que precisei e em todos os momentos mais importantes de minha vida. Eu não consigo imaginar uma mãe melhor que a minha, que se preocupou em me ligar, literalmente, todos os dias em que estive fora para saber se eu estava bem. O cuidado, os ensinamentos e o amor que eu sempre recebi nas suas atitudes me deram força para enfrentar os desafios que me foram apresentados.

  \item Ao meu irmão, Wilsin, que cresceu comigo e a quem desejo tudo de melhor que essa vida pode dar. Apesar de ser meu irmão mais novo, sempre o vi como um exemplo e o admirei pela pessoa bondosa que é, de coração grande e a quem sempre confiei. São pelos gestos, pela simples presença e pela forma sutil de mostrar que se importa que o torna um grande amigo, além de irmão.

  \item Aos meus avós, tios, primos e minha irmã Driele, que estiveram sempre comigo, me apoiando e me incentivando a ser uma pessoa melhor. Com eles aprendi muito sobre a vida e sobre como passar por ela com dignidade e respeito. Sempre me senti bem, seguro e feliz em suas companhias. Deles, sempre recebi muito amor, sorrisos e abraços e isso sempre foi muito importante e essencial para mim.

  \item À minha namorada Andreza, que esteve junto desde antes da universidade. Obrigado por dividir sua vida comigo, por estar presente todos os dias e em todas as fases, boas ou ruins. Obrigado por me fazer sentir especial, por ter sempre me respeitado e amado assim como eu a respeito e amo. Obrigado por me dar forças e conselhos quando eu precisei, por ter sido paciente em momentos de distância que vivemos pelas circunstâncias da vida e por nunca ter desistido de nós. Passar por isso junto contigo tornou tudo muito mais leve e especial. Te amo!

  \item Em extensão, agradeço à sua família, que sempre me acolheu e, em especial, à minha sogra Alzenir, que me tem como seu próprio filho. Não poderia desejar mais. Obrigado por tudo.
  
  \item Aos meus amigos e amigas, em especial, Jeferson Reis, Ítalo Oliveira, Samuel Matos, Daniel Alves, Vinícius Silva, Tayná Arruda, Débora Azevedo, Renata Barros, Endrick Ramos, Lázaro Henrique, Ana Rute, Gabriel Freitas e Jadson Araújo. Com eles eu aprendi, brinquei, ri, chorei e me diverti. Dividi tanto momentos incríveis e que levarei para sempre em minha vida. Durante as aulas, nos estudos, no descanso comendo e tomando um café na cantina ou em algum banco nós dividimos nossos pensamentos, ideias e experiências.

  \item À Include Engenharia e ao MEJ, que me fizeram crescer muito como pessoa e profissional, ao me dar a possibilidade de vivenciar grandes experiências em projetos e eventos, rodeado de pessoas incríveis.

  \item Aos meus professores e colegas de trabalho, que me acompanharam e tornaram possível a minha formação acadêmica e profissional, meu eterno agradecimento. Em especial, ao professor Orivaldo Santana, a quem admiro, respeito e que me fez decidir pelo caminho da Computação pela forma inspiradora como ele me apresentou essa área; e às professoras Raquel Sampaio e Kellen Lima, que me deram a oportunidade de vivenciar a docência por meio da monitoria de Probabilidade e Estatística, onde aprendi muito sobre como passar conhecimento e aprender com isso e sobre responsabilidade e ética.

  \item Ao meu orientador, que sempre trouxe ideias e tranquilidade nas conversas que tivemos durante o período de desenvolvimento desse trabalho.

  \item À Yasmim Araújo, responsável pelo design da aplicação desenvolvida e pelas ideias que contribuíram para o sucesso do projeto.

  \item À universidade, por tornar possível a mudança de vida que todos nós sonhamos. Sem dúvidas, ela mudou a minha e espero mudar a de outras pessoas a partir disso.

\end{trivlist}
