%
% ********** Resumo
%

% Usa-se \chapter*, e não \chapter, porque este "capítulo" não deve
% ser numerado.
% Na maioria das vezes, ao invés dos comandos LaTeX \chapter e \chapter*,
% deve-se usar as nossas versões definidas no arquivo comandos.tex,
% \mychapter e \mychapterast. Isto porque os comandos LaTeX têm um erro
% que faz com que eles sempre coloquem o número da página no rodapé na
% primeira página do capítulo, mesmo que o estilo que estejamos usando
% para numeração seja outro.
\mychapterast{Resumo}

Com o surgimento da maior pandemia já registrada, Covid-19, uma grande corrida para produção de uma vacina iniciou-se no mundo. No Brasil, as vacinações foram iniciadas no começo de 2021 a partir de um plano nacional elaborado pelo Ministério da Saúde, o qual abordava entre outras questões, a utilização de uma plataforma \textit{online} nacional para registro e controle nominal das pessoas vacinadas em todo o território brasileiro. Dentro dos cenários previstos para o registro desses dados, aquele que define o uso de formulários em papel e posterior digitação dos dados no sistema apresenta uma série de preocupações relacionadas à integridade dos dados e perda de produtividade pelos profissionais da saúde. O objetivo desse trabalho é desenvolver uma aplicação \textit{mobile} utilizando o \textit{framework Flutter}, que seja capaz de gerenciar a criação, atualização e exportação dos dados relativos à vacinação e, dessa forma, ter como resultados a minimização dos riscos relacionados ao cenário supracitado, assim como o aumento da produtividade dos profissionais da saúde responsáveis por esses registros.

% Resumo 2 baseado na descrição do trabalho
% Durante o período de pandemia houve um aumento significativo na preocupação com a integridade e monitoramento constante dos dados relacionados à vacinação. Atualmente, muitos profissionais da saúde preenchem esses dados em planilhas impressas e, posteriormente, os reescrevem em uma plataforma nacional online. Apesar da possibilidade de preenchimento diretamente no portal, muitos locais de aplicação das vacinas não fornecem uma rede de internet estável. Busca-se, portanto, desenvolver uma aplicação mobile que possa substituir as planilhas utilizadas atualmente. O desenvolvimento da aplicação mobile, utilizando-se o framework Flutter, consiste na criação de ferramentas que permitirão ao usuário cadastrar, editar, deletar e atualizar dados relacionados à vacinação de uma pessoa. Estas informações deverão ser armazenadas localmente no dispositivo móvel do usuário e recuperadas a qualquer momento para que sejam exportadas em formato de planilha (preferencialmente .xlsx) e enviadas à plataforma do SI-PNI (Sistema de Informações do Programa Nacional de Imunizações), do Ministério da Saúde. Espera-se, com esta aplicação, aumentar a produtividade desses profissionais da saúde, principalmente nas tarefas realizadas após o período de vacinação e garantir a consistência e integridade desses dados, que poderão ser cadastrados de forma offline e enviados em outro momento à plataforma online

\vspace{1.5ex}

{\bf Palavras-chave}: Flutter, Aplicação Móvel, Vacinação.

%
% ********** Abstract
%
\mychapterast{Abstract}

With the emergence of the biggest pandemic ever recorded, Covid-19, a great race to produce a vaccine began in the world. In Brazil, vaccinations began in early 2021 based on a national plan prepared by the Ministry of Health, which addressed, among other issues, the use of a national online platform for the registration and nominal control of vaccinated people throughout the Brazilian territory. Within the scenarios envisaged for recording these data, the one that defines the use of paper forms and subsequent entry of data into the system presents a series of concerns related to data integrity and loss of productivity by health professionals. The objective of this work is to develop a mobile application using the Flutter framework, which is capable of managing the creation, updating and export of data related to vaccination and, thus, having as a result the minimization of risks related to the aforementioned scenario, as well as the increase productivity of health professionals responsible for these records.

\vspace{1.5ex}

{\bf Keywords}: Flutter, Mobile Application, Vaccination.
