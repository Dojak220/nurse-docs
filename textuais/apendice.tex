%%
%% Apêndice 1: Requisitos funcionais detalhados
%%

\mychapter{Requisitos funcionais detalhados}
\label{apendice:requisitos_funcionais_detalhados}

\rowcolors{1}{white}{green!20}
\begin{longtable}{
  | >{\centering}m{0.10\textwidth} 
  | >{\centering}m{0.30\textwidth} 
  | m{0.60\textwidth} |}
  
    \hline
    \rowcolor{green!100}
    \textbf{Código} & \textbf{Requisito} & \textbf{Descrição} \\ \hline \hline
    \textbf{RF01}  &  \textbf{Cadastrar entidades} & \textbf{Permitir que o usuário cadastre novas entidades}  \\ \hline \hline
    RF01-01 & Cadastrar Vacinações              & Fornecer para preenchimento os campos obrigatórios aplicante, estabelecimento de aplicação, vacina, lote, informações do paciente (pré-cadastrado ou não), dose e data da aplicação da vacina e os campos opcionais campanha da vacinação que está sendo efetuada e data da próxima vacina \\ \hline
    RF01-02 & Cadastrar Campanhas               & Fornecer para preenchimento os campos obrigatórios título e datas de início de fim e o campo opcional descrição da campanha  \\ \hline
    RF01-03 & Cadastrar Estabelecimentos        & Fornecer para preenchimento os campos obrigatórios nome, CNES e endereço do estabelecimento de saúde \\ \hline
    RF01-04 & Cadastrar Vacinas                 & Fornecer para preenchimento os campos obrigatórios nome, fabricante e código da vacina \\ \hline
    RF01-05 & Cadastrar Lotes de Vacina         & Fornecer para preenchimento os campos obrigatórios código, nome e quantidade de doses da vacina no lote \\ \hline
    RF01-06 & Cadastrar Pacientes               & Fornecer para preenchimento os campos obrigatórios CNS, CPF, nome, data de nascimento, localidade, categoria prioritária e condição maternal e os campos opcionais sexo e nomes do pai e da mãe do paciente \\ \hline
    RF01-07 & Cadastrar Aplicantes              & Fornecer para preenchimento os campos obrigatórios CNS, CPF, nome, localidade e estabelecimento de saúde de atuação e os campos opcionais data de nascimento, sexo e nomes do pai e da mãe do paciente \\ \hline
    RF01-08 & Cadastrar Localidades             & Fornecer para preenchimento os campos obrigatórios código do IBGE, nome, cidade e estado da localidade \\ \hline
    RF01-09 & Cadastrar Grupos Prioritários     & Fornecer para preenchimento o campo obrigatório código e os campos opcionais nome e descrição do grupo prioritário \\ \hline
    RF01-10 & Cadastrar Categorias Prioritárias & Fornecer para preenchimento os campos obrigatórios* código e grupo prioritário pertencente e os campos opcionais nome e descrição da categoria prioritária \\ \hline
    RF01-11 & Botão de salvamento               & Fornecer para preenchimento os campos obrigatórios* botão para salvamento dos dados cadastrados em todos os formulários \\ \hline

  \hiderowcolors
  \caption{Requisitos funcionais da aplicação Nurse: detalhes do requisito RF01}
  \label{tab:rf01_detalhe}
\end{longtable}

% \begin{table}[ht!]
%   \centering
%   {\rowcolors{0}{white}{green!20}
%   \begin{tabularx}{\textwidth}{
%     | >{\centering\arraybackslash}m{0.10\textwidth} 
%     | >{\centering\arraybackslash}X 
%     | >{\raggedright\arraybackslash}X | }
%     \hline
%     \rowcolor{green!100}
%     \textbf{Código} & \textbf{Requisito} & \textbf{Descrição} \\ \hline \hline
%     \textbf{RF01}  &  \textbf{Cadastrar entidades} & \textbf{Permitir que o usuário cadastre novas entidades}  \\ \hline \hline
%     RF01-01 & Cadastrar Vacinações              & Fornecer para preenchimento os campos obrigatórios aplicante, estabelecimento de aplicação, vacina, lote, informações do paciente (pré-cadastrado ou não), dose e data da aplicação da vacina e os campos opcionais campanha da vacinação que está sendo efetuada e data da próxima vacina \\ \hline
%     RF01-02 & Cadastrar Campanhas               & Fornecer para preenchimento os campos obrigatórios título e datas de início de fim e o campo opcional descrição da campanha  \\ \hline
%     RF01-03 & Cadastrar Estabelecimentos        & Fornecer para preenchimento os campos obrigatórios nome, CNES e endereço do estabelecimento de saúde \\ \hline
%     RF01-04 & Cadastrar Vacinas                 & Fornecer para preenchimento os campos obrigatórios nome, fabricante e código da vacina \\ \hline
%     RF01-05 & Cadastrar Lotes de Vacina         & Fornecer para preenchimento os campos obrigatórios código, nome e quantidade de doses da vacina no lote \\ \hline
%     RF01-06 & Cadastrar Pacientes               & Fornecer para preenchimento os campos obrigatórios CNS, CPF, nome, data de nascimento, localidade, categoria prioritária e condição maternal e os campos opcionais sexo e nomes do pai e da mãe do paciente \\ \hline
%     RF01-07 & Cadastrar Aplicantes              & Fornecer para preenchimento os campos obrigatórios CNS, CPF, nome, localidade e estabelecimento de saúde de atuação e os campos opcionais data de nascimento, sexo e nomes do pai e da mãe do paciente \\ \hline
%     RF01-08 & Cadastrar Localidades             & Fornecer para preenchimento os campos obrigatórios código do IBGE, nome, cidade e estado da localidade \\ \hline
%     RF01-09 & Cadastrar Grupos Prioritários     & Fornecer para preenchimento o campo obrigatório código e os campos opcionais nome e descrição do grupo prioritário \\ \hline
%     RF01-10 & Cadastrar Categorias Prioritárias & Fornecer para preenchimento os campos obrigatórios* código e grupo prioritário pertencente e os campos opcionais nome e descrição da categoria prioritária \\ \hline
%     RF01-11 & Botão de salvamento               & Fornecer para preenchimento os campos obrigatórios* botão para salvamento dos dados cadastrados em todos os formulários \\ \hline
%   \end{tabularx}}
% \caption{Requisitos funcionais da aplicação Nurse: detalhes do requisito RF01}
% \label{tab:rf01_detalhe}
% \end{table}

\begin{table}[ht!]
  \centering
  {\rowcolors{0}{white}{green!20}
  \begin{tabularx}{\textwidth}{
    | >{\centering\arraybackslash}m{0.10\textwidth} 
    | >{\centering\arraybackslash}m{0.30\textwidth} 
    | >{\raggedright\arraybackslash}X | }
    \hline
    \rowcolor{green!100}
    \textbf{Código} & \textbf{Requisito} & \textbf{Descrição} \\ \hline \hline
    \textbf{RF02}  &  \textbf{Visualizar entidades cadastradas} & \textbf{Permitir que o usuário visualize as entidades cadastradas e seus detalhes}  \\ \hline \hline
    RF02-01 & Visualizar Vacinações               & Apresentar CNS, nome e grupo do paciente e vacina aplicada \\ \hline
    RF02-02 & Visualizar Campanhas                & Apresentar título, datas de início de fim e status da campanha \\ \hline
    RF02-03 & Visualizar Estabelecimentos         & Apresentar nome, CNES e endereço do estabelecimento de saúde \\ \hline
    RF02-04 & Visualizar Vacinas                  & Apresentar nome, fabricante e código da vacina \\ \hline
    RF02-05 & Visualizar Lotes de Vacina          & Apresentar código, nome e quantidade de doses da vacina no lote \\ \hline
    RF02-06 & Visualizar Pacientes                & Apresentar CNS, nome, categoria prioritária e condição maternal do paciente \\ \hline
    RF02-07 & Visualizar Aplicantes               & Apresentar CNS, nome e estabelecimento de saúde do profissional \\ \hline
    RF02-08 & Visualizar Localidades              & Apresentar código do IBGE, nome e endereço da localidade \\ \hline
    RF02-09 & Visualizar Grupos Prioritários      & Apresentar código, nome e descrição do grupo prioritário \\ \hline
    RF02-10 & Visualizar Categorias Prioritárias  & Apresentar código, nome e descrição da categoria prioritária \\ \hline
  \end{tabularx}}
\caption{Requisitos funcionais da aplicação Nurse: detalhes do requisito RF02}
\label{tab:rf02_detalhe}
\end{table}

\begin{table}[ht!]
  \centering
  {\rowcolors{0}{white}{green!20}
  \begin{tabularx}{\textwidth}{
    | >{\centering\arraybackslash}m{0.10\textwidth} 
    | >{\centering\arraybackslash}m{0.30\textwidth} 
    | >{\raggedright\arraybackslash}X | }
    \hline
    \rowcolor{green!100}
    \textbf{Código} & \textbf{Requisito} & \textbf{Descrição} \\ \hline \hline
    \textbf{RF04}  &  \textbf{Gerar tabela de vacinações} & \textbf{Permitir que o usuário gere uma
    tabela de vacinações para
    exportação}  \\ \hline \hline
    RF04-01 & Escolher período de tempo               & Permitir que o usuário escolha o período desejado para coleta dos dados sobre vacinação a serem exportados \\ \hline
    RF04-02 & Escolher o que fazer com o arquivo gerado                     & Permitir que o usuário escolha entre abrir o arquivo com a planilha gerada ou exportá-la utilizando para isso alguma aplicação externa compatível com a escolha, a depender do caso \\ \hline
  \end{tabularx}}
\caption{Requisitos funcionais da aplicação Nurse: detalhes do requisito RF04}
\label{tab:rf04_detalhe}
\end{table}