%%
%% Capítulo 4: Problema
%%

\mychapter{Problema}
\label{Cap:Problema}

% Planilha de registro manual de vacinados. Essa imagem deve ser colocada para explicar como é feita a vacinação atualmente.
% Registro Manual de Vacinados - Conasemshttps://www.conasems.org.br 02/11/2022

% [A fazer] Tirar print da plataforma online de cadastro das vacinações
\section{Procedimento comum de vacinação}
\label{Sec:ProcedimentoComumVacinacao}
O material de trabalho comumente utilizado é o papel com a planilha para preenchimento impressa. Feitas as anotações, esses dados são digitados em um sistema de cadastro de vacinação. Dessa forma, existe alguns pontos de falha no processo, como a possibilidade de erro de escrita da informação no momento da sua coleta ou de digitação, no momento em que esses dados são repassados para o sistema online; perda dos dados e dificuldade de compartilhamento dos mesmos.

% [A fazer] Adicionar imagem da planilha disponibilizada pelo governo para o registro escrito de vacinados
% [A fazer] Adicionar print da plataforma online de cadastro das vacinações

% [A analisar] Posso ir me perguntando vários porquês das coisas e colocar aqui para escrever sobre isso.

% [A analisar] Posso juntar trabalhos relacionados com problema em um mesmo capítulo.


% \begin{table}[htbp]
% \begin{tabularx}{\linewidth}{|p{3cm}|X|l|} \hline
% COLUNA p & COLUNA X & COLUNA l \\ \hline
% Largura fixa (não depende do conteúdo) &
% Expandível &
% Ajustável \\ \hline
% Alinhada no topo &
% Alinhada à esquerda &
% Alinhada à esquerda \\ \hline
% \end{tabularx}
% \\[0.5cm]
% \begin{tabularx}{\linewidth}{|b{3cm}|C|r|} \hline
% COLUNA b & COLUNA C (ver \texttt{comandos.tex}) & COLUNA r \\ \hline
% Largura fixa (não depende do conteúdo) &
% Expandível &
% Ajustável \\ \hline
% Alinhada na base &
% Centralizada &
% Alinhada à direita \\ \hline
% \end{tabularx}
% \caption{Tabelas com colunas de diferentes larguras e alinhamentos}
% \label{Tab:larguracolunas}
% \end{table}

