%%
%% Capítulo 2: Regras gerais de estilo
%%

\mychapter{Fundamentação Teórica}
\label{Cap:Teoria}

Neste capítulo serão apresentados os conceitos básicos relacionados ao \textit{framework Flutter}, assim como a linguagem de programação Dart, a biblioteca MobX e o \textit{wrapper} Provider, utilizados para o desenvolvimento desta aplicação.

Também serão discutidas as principais ideias relacionadas à Programação Orientada a Objeto (POO), aos bancos de dados relacionais e, por fim, alguns conceitos relacionados às boas práticas da programação no quesito de arquitetura de sistemas e os princípios SOLID.

\section[\textit{Flutter}]{Flutter}

Busca-se apresentar, nesta seção, uma visão em alto nível do que é o \textit{framework Flutter}, sua arquitetura, as principais características da tecnologia e suas diferenças para as demais tecnologias no mercado.

O \textit{Flutter}, criado pela \textit{Google}, é desenvolvido em código aberto e visa possibilitar o desenvolvimento de aplicações \textit{Android}, \textit{Web}, \textit{Desktop} e de \textit{software} embarcado a partir de uma única base de código, compiladas nativamente. A aplicação é mantida não só pela empresa que a criou, mas também recebe novas atualizações e incorporações advindas da comunidade.
% ref: https://docs.flutter.dev/resources/faq#what-is-flutter

\subsection{Tipos de \textit{Widgets}}

\subsubsection{Stateless Widgets}

\subsection{Stateful Widgets}
