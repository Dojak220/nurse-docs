%%
%% Apêndice 2: Pacotes e versões utilizados
%%

\mychapter{Pacotes e versões utilizados}
\label{apendice:pacotes}

\rowcolors{1}{white}{green!10}
\begin{longtable}{
  | >{\centering}m{0.30\textwidth} 
  | >{\centering}m{0.20\textwidth} 
  | m{0.50\textwidth} |}
  
    \hline
    \rowcolor{green!100}
    \textbf{Pacote} & \textbf{Versão} & \textbf{Descrição} \\ \hline \hline
    cupertino\_icons           & \^{} 1.0.2       & Repositório de ícones utilizados pelos \textit{widget}s do Cupertino \cite{cupertino-package}  \\ \hline
    intl                       & \^{} 0.17.0       & Repositório utilizado para formatação de datas nessa aplicação \cite{intl-package}  \\ \hline
    sqflite\_sqlcipher         & \^{} 2.1.0        & Extensão ao \textit{sqflite} \cite{sqflite-package} que adiciona senha ao acesso o banco de dados \cite{sqlcipher-package} \\ \hline
    path                       & \^{} 1.8.0        & Biblioteca para manipulação de caminhos em multi-plataformas \cite{path-package}               \\ \hline
    provider                   & \^{} 6.0.3        & Encapsula o \textit{InheritedWidget}, tornando-o reutilizável e mais fácil de usar \cite{provider-package}\\ \hline
    flutter\_archive           & \^{} 5.0.0        & Biblioteca para criação e extração de arquivos ZIP \cite{flutter_archive-package}              \\ \hline
    flutter\_dotenv            & \^{} 5.0.2        & Carrega configurações para aplicação em tempo de execução \cite{flutter_dotenv-package}        \\ \hline
    flutter\_mobx              & \^{} 2.0.5        & Integração do MobX aos \textit{widget}s do \textit{Flutter}                                    \\ \hline
    mobx                       & \^{} 2.0.7        & Biblioteca para gerenciamento de estado na aplicação                                           \\ \hline
    syncfusion\_flutter\_xlsio & \^{} 20.2.50-beta & Pacote para criação de arquivos Excel (.xlsx) \cite{syncfusion_flutter_xlsio-package}          \\ \hline
    path\_provider             & \^{} 2.0.11       & \textit{Plugin} para encontrar locais no sistema em múltiplas plataformas \cite{path_provider-package}  \\ \hline
    open\_file                 & \^{} 3.2.1        & \textit{Plugin} para abertura de arquivos do sistema em múltiplas plataformas \cite{open_file-package} \\ \hline
    share\_plus               & \^{} 4.4.0        & \textit{Plugin} para compartilhamento de arquivos a partir da aplicação \cite{share_plus-package}       \\
    \hline
  \hiderowcolors
  \caption{Pacotes utilizados no projeto e suas respectivas versões}
  \label{tab:packages}
\end{longtable}