%%
%% Capítulo 4: Nurse: uma aplicação para produtividade em vacinações
%%

\mychapter{Nurse: uma aplicação para produtividade em vacinações}
\label{Cap:Implementacao}
\section{Arquitetura do Sistema}
\label{cap4:SubSec:ArquiteturaSistema}

\section{Telas}
\label{cap4:Sec:Telas}

\section{Arquitetura do Sistema}
\label{Sec:ArquiteturaSistema}

\section{Persistência de Dados}
\label{Sec:PersistenciaDados}

\subsection{Diagrama de Classes}
\label{Sec:DiagramaClasses}

\subsection{Uso do Banco de Dados}
\label{Sec:UsoBancoDados}

\section{Pacotes e Bibliotecas}
\label{Sec:PacotesBibliotecas}

\subsection{MobX}
\label{Subsec:MobX}

\subsection{Provider}
\label{Subsec:Provider}

\subsection{excel}
\label{Subsec:excel}



















% \begin{algorithm}
% %% \SetLine
% \Entrada{$x$: vetores de valores; $y$ = $L(x)$; $p$: valor de entrada a ser calculado }
% \Saida{$s$ = $L(p)$}
% $n \leftarrow \mathtt{comprimento}(x)$\;
% $s \leftarrow 0$\;
% \Para {$i=1$ \Ate $n$} {
% 	$L \leftarrow 1$\;
% 	\Para {$j=1:1:n$} {
% 		\Se{$i \neq j$} {
% 			$L \leftarrow L* \left( \dfrac{p-x[j]}{x[i]-x[j]} \right) $
% 		}
% 	}
% 	$s \leftarrow s + L*y[i]$\;
% }
% \Retorna $s$\;
% \caption{Algoritmo para interpolação de Lagrange.}
% \label{algo:1}
% \end{algorithm}

% \begin{algorithm}
% %% \SetLine
% \Entrada{$a$: valor inicial; $b$: valor final; $n$: número de subintervalos (deve ser múltiplo de 2)  }
% \tcc{A função a ser integrada é definida em uma função denominada \texttt{f}, fora do escopo deste algoritmo.}
% \Saida{$I$ = integral de \texttt{f} entre $a$ e $b$}
% $h \leftarrow$ $\dfrac{b-a}{n}$\;
% $x[1] \leftarrow a$\;
% $y[1] \leftarrow f(a)$\;
% $I \leftarrow 0$\;
% $k \leftarrow 2$\;
% \Enqto {$k <= n$} {
% 	$x[i] \leftarrow x[i-1] + h$\;
% 	$y[i] \leftarrow f(x[i])$\;
% 	\eSe{$i \% 2 = 0$} {
% 		$I \leftarrow I + 4*y[i]$\;
% 	}
% 	{
% 		$I \leftarrow I + 2*y[i]$\;
% 	}
% 	$k = k+1$\;
% }
% $x[n+1] \leftarrow b$\;
% $y[n+1] \leftarrow f(x[i+1])$\;
% $I \leftarrow I + \dfrac{h}{3}*(I + y[n+1])$\;
% \Retorna $I$\;
% \caption{Algoritmo para a integração pelo primeiro método de Simpson.}
% \label{algo:2}
% \end{algorithm}
