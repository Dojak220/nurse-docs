%%
%% Capítulo 1: Introdução
%%

\mychapter{Introdução}
\label{Cap:Introducao}
% Dicas para o trabalho
% [A revisar] Fazer uma ou duas perguntas relacionadas ao trabalho que serão respondidas no decorrer do mesmo.
% O objeto do projeto é tornar o registro de dados sobre a vacinação contra a COVID-19 mais fácil e rápido para os profissionais de saúde, de forma que possam se concentrar em outras atividades. Esse registro é feito manualmente e em papel, o que pode gerar erros e atrasos no processo de cadastro. O objetivo do projeto é criar um sistema que organize os dados e envio-os por meio de planilhas que poderão ser enviadas ao sistema único de saúde.

% [A fazer] Escrever por que essa solução deve ser usada e qual o problema ele resolve.

% [As informações de Titita] Existem dois sistemas para informação das vacinações do município: esus pec e o novo sipni.
% O manual de vacinação
% O velho sipni (sipni web) era usado paras campanhas de vacinação contra influenza e poliomelite. Agora todas as campanhas são feitas no novo sipni. Nesse ano de 2022 já foram feitas campanhas contra sarampo, polio e influenza.

% Cada campanha tem um manual técnico/operacional. 


\section{Contextualização}
\label{cap1:Sec:Contextualizacao}
% [A revisar]
Em 2019 surgiu o vírus SARS-Cov-2, responsável por causar a maior pandemia já registrada na história, chamada Covid-19. Desde então, uma grande mobilização em todos os países começou visando a criação de uma vacina contra a doença causada pelo vírus e a consequente corrida para a sua produção em massa. De acordo com a Organização Mundial da Saúde (OMS), no início de 2021 existiam 236 vacinas candidatas em fases pré- clínicas ou fase clínica.
% ref.: Plano Nacional de Operacionalização da Vacinação contra a Covid-19 - PNO (2ª Edição com ISBN) 1. Introdução - https://www.gov.br/saude/pt-br/coronavirus/vacinas/plano-nacional-de-operacionalizacao-da-vacina-contra-a-covid-19


No Brasil, a vacinação contra a COVID-19 começou em 18 de janeiro de 2021, com a vacina Coronavac, produzida pelo Instituto Butantan em parceria com a farmacêutica chinesa Sinovac. A vacinação foi realizada em etapas, à medida em que as vacinas eram fabricadas e priorizando os grupos de maior risco de contaminação e de complicações da doença: profissionais da saúde, idosos, pessoas com comorbidade, pessoas com deficiência e população indígena. A vacinação foi feita em todo o país e com a utilização de um plano nacional, mas cada estado e município teve sua própria estratégia para a vacinação, em geral, pautada nesse plano.
% ref.: Plano Nacional de Operacionalização da Vacinação contra a Covid-19 - PNO (2ª Edição com ISBN) 3. População Alvo - https://www.gov.br/saude/pt-br/coronavirus/vacinas/plano-nacional-de-operacionalizacao-da-vacina-contra-a-covid-19


% [A revisar]
Independente da estratégia de vacinação de cada estado e em decorrência da necessidade de mapear com exatidão as pessoas que receberam a vacina, o Ministério da Saúde criou um módulo para o registro de dados nominais sobre a vacinação, o que inclui dados pessoais d(a) vacinado(a), e dose e lote administrados.

Esse sistema foi denominado Sistema de Informação de Programas de Imunização (Novo SI-PNI), e é usado para o registro de dados sobre as campanhas de vacinação. O SIPNI é um sistema web, que pode ser acessado pelo site da SES-SP, e é usado por profissionais de saúde para o registro de dados sobre a vacinação. O SIPNI é dividido em duas partes: o SIPNI Web e o SIPNI Mobile. O SIPNI Web é usado para o registro de dados sobre as campanhas de vacinação, e o SIPNI Mobile é usado para o registro de dados sobre a vacinação em campo.

% ref.: Plano Nacional de Operacionalização da Vacinação contra a Covid-19 - PNO (2ª Edição com ISBN) 3. População Alvo - https://www.gov.br/saude/pt-br/coronavirus/vacinas/plano-nacional-de-operacionalizacao-da-vacina-contra-a-covid-19

\section{Problema}
\label{cap1:Sec:Problema}
% Planilha de registro manual de vacinados. Essa imagem deve ser colocada para explicar como é feita a vacinação atualmente.
% Registro Manual de Vacinados - Conasems https://www.conasems.org.br 02/11/2022

% [A fazer] Tirar print da plataforma online de cadastro das vacinações
\subsection{Procedimento comum de vacinação}
\label{cap1:SubSec:cap1:ProcedimentoComumVacinacao}
O material de trabalho comumente utilizado é o papel com a planilha para preenchimento impressa. Feitas as anotações, esses dados são digitados em um sistema de cadastro de vacinação. Dessa forma, existe alguns pontos de falha no processo, como a possibilidade de erro de escrita da informação no momento da sua coleta ou de digitação, no momento em que esses dados são repassados para o sistema online; perda dos dados e dificuldade de compartilhamento dos mesmos.

% Ficha de vacinação: https://sisaps.saude.gov.br/esus/ ---> https://sisaps.saude.gov.br/esus/#:~:text=Manual%20PEC/CDS-,Fichas%20%2D%20CDS,-Fichas%20vigentes --->  http://189.28.128.100/dab/docs/portaldab/documentos/esus/ficha_vacinacao_COVID-19.pdf

% [A fazer] Adicionar imagem da planilha disponibilizada pelo governo para o registro escrito de vacinados
% [A fazer] Adicionar print da plataforma online de cadastro das vacinações

% [A analisar] Posso ir me perguntando vários porquês das coisas e colocar aqui para escrever sobre isso.

% [A analisar] Posso juntar trabalhos relacionados com problema em um mesmo capítulo.



\section{Objetivos}
\label{cap1:Sec:Objetivos}

O objetivo geral do trabalho é desenvolver a aplicação \textit{mobile} Nurse, que tem como fim facilitar o registro de dados sobre as vacinações realizadas pelos profissionais da saúde, reduzir a possibilidade de erros de escrita e de perda de dados, além de agilizar a integração desses dados com o sistema nacional de imunizações. Como objetivos específicos, tem-se:
\begin{itemize}
  \item Apresentar o contexto ao qual a aplicação se destina, destacando-se o problema que ela pretende resolver;
  \item Relatar o processo de desenvolvimento da aplicação, desde a sua concepção até a sua implementação;
  \item Apresentar os resultados obtidos com o desenvolvimento da aplicação;
\end{itemize}

\section{Estrutura do Trabalho}
\label{cap1:Sec:EstruturaTrabalho}

O trabalho está estruturado em seções, como descrito a seguir: 


  \begin{enumerate}[label=\textbf{Seção \arabic*}]
    \item \textbf{Introdução}: trata-se desta seção. Aqui são apresentados o problema, os objetivos e estrutura geral do trabalho.
    \item \textbf{Fundamentação Teórica}: são apresentados os principais conceitos relacionados à plataforma e à modelagem do banco de dados utilizados no desenvolvimento da aplicação.
    % \item \textbf{Trabalhos Relacionados}: descreve aplicações que possuem similaridades com a aplicação Nurse em relação à tecnologia utilizada e/ou área de aplicação.
    \item \textbf{Implementação}: os requisitos, casos de uso e arquitetura da aplicação Nurse, assim como o seu desenvolvimento e os pacotes utilizados para isso são apresentados nessa seção.
    \item \textbf{Experimentos e Resultados}: é nessa seção que são apresentados os testes desenvolvidos para avaliar as funcionalidades da aplicação e os seus resultados. Além disso, são apresentados os principais fluxos se uso da aplicação.
    \item \textbf{Conclusão}: seção dedica às considerações finais sobre o trabalho e as perspectivas futuras para novas funcionalidades que poderão vir a ser implementadas.
    \item \textbf{Referências}: apresenta as referências utilizadas no trabalho.
    \item \textbf{Informações Adicionais}: apresenta, em formato de apêndices, informações adicionais que complementam as informações apresentadas em outras seções do trabalho.
  \end{enumerate}



% \begin{table}[htbp]
% \begin{tabularx}{\linewidth}{|p{3cm}|X|l|} \hline
% COLUNA p & COLUNA X & COLUNA l \\ \hline
% Largura fixa (não depende do conteúdo) &
% Expansível &
% Ajustável \\ \hline
% Alinhada no topo &
% Alinhada à esquerda &
% Alinhada à esquerda \\ \hline
% \end{tabularx}
% \\[0.5cm]
% \begin{tabularx}{\linewidth}{|b{3cm}|C|r|} \hline
% COLUNA b & COLUNA C (ver \texttt{comandos.tex}) & COLUNA r \\ \hline
% Largura fixa (não depende do conteúdo) &
% Expandível &
% Ajustável \\ \hline
% Alinhada na base &
% Centralizada &
% Alinhada à direita \\ \hline
% \end{tabularx}
% \caption{Tabelas com colunas de diferentes larguras e alinhamentos}
% \label{Tab:larguracolunas}
% \end{table}